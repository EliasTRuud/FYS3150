%-------------------- begin preamble ----------------------

\documentclass[%
oneside,                 % oneside: electronic viewing, twoside: printing
final,                   % draft: marks overfull hboxes, figures with paths
10pt]{article}

\listfiles               %  print all files needed to compile this document
\usepackage{url}			% needed for citing
\usepackage{relsize,makeidx,color,setspace,amsmath,amsfonts,amssymb}
\usepackage[table]{xcolor}
\usepackage{bm,ltablex,microtype}
\newcommand{\R}{\mathbb{R}} 

\usepackage[pdftex]{graphicx}
\usepackage{blkarray} %needed for indexing matrix
\usepackage{fancyvrb} % packages needed for verbatim environments

\usepackage[T1]{fontenc}
%\usepackage[latin1]{inputenc}
\usepackage{ucs}
\usepackage[utf8x]{inputenc}

\usepackage{lmodern}         % Latin Modern fonts derived from Computer Modern

% Hyperlinks in PDF:
\definecolor{linkcolor}{rgb}{0,0,0.4}
\usepackage{hyperref}
\hypersetup{
    breaklinks=true,
    colorlinks=true,
    linkcolor=linkcolor,
    urlcolor=linkcolor,
    citecolor=black,
    filecolor=black,
    %filecolor=blue,
    pdfmenubar=true,
    pdftoolbar=true,
    bookmarksdepth=3   % Uncomment (and tweak) for PDF bookmarks with more levels than the TOC
    }
%\hyperbaseurl{}   % hyperlinks are relative to this root

\setcounter{tocdepth}{2}  % levels in table of contents


% prevent orhpans and widows
\clubpenalty = 10000
\widowpenalty = 10000

% --- end of standard preamble for documents ---


% insert custom LaTeX commands...

\raggedbottom
\makeindex
\usepackage[totoc]{idxlayout}   % for index in the toc
% \usepackage[nottoc]{tocbibind}  % for references/bibliography in the toc

% \usepackage{booktabs}				% for citing
\usepackage{float}					% for placing figures
%-------------------- end preamble ----------------------

\begin{document}




% matching end for #ifdef PREAMBLE



% ------------------- main content ----------------------



% ----------------- title -------------------------

\thispagestyle{empty}

\begin{titlepage}
   \begin{center}
       \vspace*{1cm}
       \Huge\textbf{----------}
\end{center}
\begin{center}

       \vspace{0.5cm}
       \large
        ------------
            
       \vspace{1.5cm}

       \textbf{Mira Mors and Elias Tidemand Ruud}

       \vfill

 
            
      {\bf \href{{http://www.uio.no/studier/emner/matnat/fys/FYS3150/index-eng.html}}{Computational Physics I FYS3150/FYS4150}} 
            
       \vspace{0.8cm}
     
            
       Department of Physics\\
       University of Oslo\\
       Norway\\
       September 2020
            
   \end{center}
\end{titlepage}





\tableofcontents
\newpage
\section{Abstract}
\section{Introduction}
\section{Methods}
Since $\mathbf{U}$ is orthogonal holds $\mathbf{U^TU} = I$
\begin{equation}
w_i^Tw_j = (Uv_i)^T(Uv_j)=v_i^TU^TUv_j=v_i^TIv_j = v_i^Tv_j
\end{equation}
\subsection{Jacobi Method}
The Jacobi method guarantees a solution for all real symmetric matrices. 
The method rotates the initial matrix using similarity transformations. By discarding the off-diagonal elements, the eigenvalues stay preserved.
How does this work? Firstly, we look at the transformation matrix. \\
The transformation matrix is orthogonal since its inverse is equal to its transpose
\begin{equation*}
\mathbf{S}^T = \mathbf{S}^{-1}.
\end{equation*}
In $\R^2$, the transformation matrix is given by
\[
     \begin{bmatrix}
     				cos(\theta) & -sin(\theta) \\
					sin(\theta) & cos(\theta)
                      \end{bmatrix}
\]
and $\theta$ denotes the degree of rotation in the plane. \\
In $\R^3$, the rotation along the x, y and z axis respectively are 
\[
    R_x = \begin{bmatrix}
     				1 			&	0 			&0				\\
     				0			&	cos(\theta) & -sin(\theta) \\
					0			&	sin(\theta) & cos(\theta)
                      \end{bmatrix}
	\hspace{0.5cm} R_y = \begin{bmatrix}
     				cos(\theta)&	0 			& -sin(\theta)		\\
     				0			&	1 			& 0 				\\
					sin(\theta)	&	0			& cos(\theta)
                      \end{bmatrix}
	\hspace{0.5cm} R_z = \begin{bmatrix}
     				cos(\theta)& -sin(\theta)	& 	\\
     				sin(\theta)	& cos(\theta) 	& 0 \\
						0		&	0			& 1
                      \end{bmatrix}
\]
Generally, a rotation i $\R^n$ is given by 
\[
S(k,l,\theta) = \begin{blockarray}{cccccccc}
 		&  				& l				&			&k 				& 			&  \\
\begin{block}{[ccccccc] c}
  1 	& \ldots 		&  0			&  \ldots	&  0			& \ldots	& 0 	&	 \\
  \vdots & \ddots 		& 	\vdots		& 	 		& 	\vdots		& 			& \vdots &  	 \\
  	0	& 	\ldots		& cos(\theta) 	& \ldots 	& -sin(\theta)	& 	\ldots	& 0		& l	 \\
  \vdots&  				& \vdots 		& \ddots 	& \vdots		& 			&\vdots&	 \\
    0	& 	 \ldots		& sin(\theta)	& \ldots	& cos(\theta) 	& 	\ldots	&	0	& k	\\
  \vdots &  			& \vdots		& 			& 	\vdots		& \ddots	& \vdots & 	\\
  0 & \ldots			& 0	 			& \ldots	& 0				& \ldots	& 1 &  	\\
\end{block}
\end{blockarray}
 \]
$cos(\theta)$ and $sin(\theta)$ are now placed in the l\textsuperscript{th} row and l\textsuperscript{th} column and k\textsuperscript{th} row and k\textsuperscript{th} column. \\
The similarity transformation 
\begin{equation}
\mathbf{A}_{new} = \mathbf{S}^T(k,l,\theta) \cdot \mathbf{A} \cdot \mathbf{S}(k,l,\theta)
\end{equation}
rotates row and column k and l of $\mathbf{A}$ an angle $\theta$ such as the entries $\mathbf{A}_  {new}(k,l)$ and $ \mathbf{A}_{new}(l,k)$ become zero.

The new entries are given with
\begin{equation}
\begin{aligned}
b_{ii} &= a_{ii} \hspace{0.5cm} i \neq k, i \neq l \\
b_{ik} &= a_{ik}cos(\theta) - a_{il}sin(\theta) \hspace{0.5cm} i \neq k, i \neq l \\
b_{il} &= a_{il}cos(\theta) + a_{ik}sin(\theta) \hspace{0.5cm} i \neq k, i \neq l\\
b_{kk} &= a_{kk}cos^2(\theta) -2a_{kl}cos(\theta)sin(\theta) + a_llsin^2(\theta)\\
b_{ll} &= a_{ll}cos^2(\theta) + 2a_{kl}cos(\theta)sin(\theta) + a_{kk}sin^2(\theta)\\
b_{kl} &= (a_{kk}-a_{ll})cos(\theta)sin(\theta) + a_{kl}(cos^2(\theta)-sin^2(\theta))
\end{aligned}
\end{equation}
The aim is to get zero for all non-diagonal elements $b_{kl}$. For each iteration, $\theta$ must be chosen accordingly. 
The Jacobi Method is an iterative method, thus this procedure is continued until the sum over the squared non-diagonal matrix elements, $off(\mathbf{A})$, are less than a prefixed test, $\epsilon$, (ideally equal zero). That is
\begin{equation}
off(\mathbf{A}) = \sqrt{\sum_{i=1}^{n} \sum_{j=1, j \neq i}^{n} = |a_{ij}|^2} < \epsilon
\end{equation}
Since this is quite a time-consuming test, it can be replaced by finding the largest (absolute) value of the off-diagonal elements
\begin{equation}
max|a_{ij}| < \epsilon \hspace{0.5cm} i \neq j.
\end{equation}
Even though $\mathbf{A}_  {new}(k,l)$ and $ \mathbf{A}_{new}(l,k)$ are set to zero for one iteration, these values can 
Given a real symmetric matrix, the algorithm is set to succeed. However, the large amount of iterations (about $n^3$) make the method very slow.


%https://www.patnauniversity.ac.in/e-content/science/physics/MScPhy58.pdf

\end{document}