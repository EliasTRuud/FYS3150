%%
%% Automatically generated file from DocOnce source
%% (https://github.com/hplgit/doconce/)
%%
%%


%-------------------- begin preamble ----------------------

\documentclass[%
oneside,                 % oneside: electronic viewing, twoside: printing
final,                   % draft: marks overfull hboxes, figures with paths
10pt]{article}

\listfiles               %  print all files needed to compile this document

\usepackage{relsize,makeidx,color,setspace,amsmath,amsfonts,amssymb}
\usepackage[table]{xcolor}
\usepackage{bm,ltablex,microtype}
\newcommand{\R}{\mathbb{R}}

\usepackage[pdftex]{graphicx}

\usepackage{fancyvrb} % packages needed for verbatim environments

\usepackage[T1]{fontenc}
%\usepackage[latin1]{inputenc}
\usepackage{ucs}
\usepackage[utf8x]{inputenc}

\usepackage{lmodern}         % Latin Modern fonts derived from Computer Modern

% Hyperlinks in PDF:
\definecolor{linkcolor}{rgb}{0,0,0.4}
\usepackage{hyperref}
\hypersetup{
    breaklinks=true,
    colorlinks=true,
    linkcolor=linkcolor,
    urlcolor=linkcolor,
    citecolor=black,
    filecolor=black,
    %filecolor=blue,
    pdfmenubar=true,
    pdftoolbar=true,
    bookmarksdepth=3   % Uncomment (and tweak) for PDF bookmarks with more levels than the TOC
    }
%\hyperbaseurl{}   % hyperlinks are relative to this root

\setcounter{tocdepth}{2}  % levels in table of contents

% --- fancyhdr package for fancy headers ---
\usepackage{fancyhdr}
\fancyhf{} % sets both header and footer to nothing
\renewcommand{\headrulewidth}{0pt}
\fancyfoot[LE,RO]{\thepage}
% Ensure copyright on titlepage (article style) and chapter pages (book style)
\fancypagestyle{plain}{
  \fancyhf{}
  %\fancyfoot[C]{{\footnotesize \copyright\ 1999-2020, "Computational Physics I FYS3150/FYS4150":"http://www.uio.no/studier/emner/matnat/fys/FYS3150/index-eng.html". Released under CC Attribution-NonCommercial 4.0 license}}
%  \renewcommand{\footrulewidth}{0mm}
  \renewcommand{\headrulewidth}{0mm}
}
% Ensure copyright on titlepages with \thispagestyle{empty}
\fancypagestyle{empty}{
  \fancyhf{}
  %\fancyfoot[C]{{\footnotesize \copyright\ 1999-2020, "Computational Physics I FYS3150/FYS4150":"http://www.uio.no/studier/emner/matnat/fys/FYS3150/index-eng.html". Released under CC Attribution-NonCommercial 4.0 license}}
  \renewcommand{\footrulewidth}{0mm}
  \renewcommand{\headrulewidth}{0mm}
}

\pagestyle{fancy}


% prevent orhpans and widows
\clubpenalty = 10000
\widowpenalty = 10000

% --- end of standard preamble for documents ---


% insert custom LaTeX commands...

\raggedbottom
\makeindex
\usepackage[totoc]{idxlayout}   % for index in the toc
\usepackage[nottoc]{tocbibind}  % for references/bibliography in the toc

%-------------------- end preamble ----------------------

\begin{document}

% matching end for #ifdef PREAMBLE

\newcommand{\exercisesection}[1]{\subsection*{#1}}


% ------------------- main content ----------------------



% ----------------- title -------------------------

\thispagestyle{empty}

\begin{center}
{\LARGE\bf
\begin{spacing}{1.25}
Project 1
\end{spacing}
}
\end{center}

% ----------------- author(s) -------------------------

\begin{center}
{\bf \href{{http://www.uio.no/studier/emner/matnat/fys/FYS3150/index-eng.html}}{Computational Physics I FYS3150/FYS4150}}
\end{center}

    \begin{center}
% List of all institutions:
\centerline{{\small Department of Physics, University of Oslo, Norway}}
\end{center}
    
% ----------------- end author(s) -------------------------

% --- begin date ---
\begin{center}
Aug 25, 2020
\end{center}
% --- end date ---

\vspace{1cm}

%\subsection*{Abstract}
%\subsection*{Introduction}
%\subsection*{Method}
\section{Abstract}
\section{Introduction}
\section{Method}
\paragraph{Approximation of the second derivative}
Given the one-dimensional Poisson equation with Dirichlet boundary conditions
\begin{equation*}
-u''(x) = f(x), \hspace{0.5cm} x\in(0,1), \hspace{0.5cm} u(0) = u(1) = 0.
\end{equation*}
we approximate the second derivative of $u$ with
\begin{equation}\label{eqn:2ndDeriv}
   -\frac{v_{i+1}+v_{i-1}-2v_i}{h^2} = f_i  \hspace{0.5cm} \mathrm{for} \hspace{0.1cm} i=1,\dots, n,
\end{equation}
where $f_i=f(x_i)$ and the discretized approximation to $u$ is $v_i$ with grid points $x_i = ih$ in the interval from $x_0$ to $x_{n+1}=1$.
Further, the step length, h,  is defined as $h=1/(n+1)$ and the boundary conditions set $v_0 = v_{n+1}=0$. \\
For numerical computation, (\ref{eqn:2ndDeriv}) must be written as a linear set of equations of the form
\begin{equation*}
   \mathbf{A}\mathbf{v} = \tilde{\mathbf{g}},
\end{equation*}
where $\mathbf{A}$ is an $n\times n$  tridiagonal matrix.\\
The set of equation for $h^2f_i$ is given by
\begin{equation*}
   -v_{i+1}-v_{i-1}+2v_i = h^2f_i  \hspace{0.5cm} 
\end{equation*}
Rearranging the equation gives
\begin{equation*}
  -v_{i-1}+2v_i - v_{i+1}= h^2f_i  \hspace{0.5cm} 
\end{equation*}
Determining the equations for the boundary condition, $i=1$ holds 
\begin{equation*}
\begin{aligned}
  -v_{0}+2v_1 - v_{2}&= h^2f_1  \hspace{0.5cm} \\
  0 + 2v_1 - v_{2}&= h^2f_1  \hspace{0.5cm} 
\end{aligned}
\end{equation*}

since $v_0=0$. Given $v_{n+1}=0$, similarly for $i=n$ applies
\begin{equation*}
\begin{aligned}
  -v_{n-1}+2vn - v_{n+1}&= h^2f_n  \hspace{0.5cm} \\
 -v_{n-1}+ 2v_n - 0&= h^2f_n  \hspace{0.5cm} 
\end{aligned}
\end{equation*}
A combination of the above equations results in 
\begin{equation*}
\begin{aligned}
  2v_1 - v_{2}&= h^2f_1 & = \tilde{g_1}  \hspace{0.5cm} \\
  ...\\
  -v_{i-1}+2v_i - v_{i+1}&= h^2f_i &= \tilde{g_i}  \hspace{0.5cm} \\
  ...\\
 -v_{n-1}+ 2v_n &= h^2f_n &= \tilde{g_n}  \hspace{0.5cm} 
\end{aligned}
\end{equation*}
Thus, there are $n$ equations that must be computed ($i \in [1,n])$.
The set of equations can be rewritten in matrix-form, there $ \mathbf{A} \in \R^{n \times n} $ and $ \mathbf{v}, \mathbf{g} \in \R^{n}$
\[
     \begin{bmatrix}
                           2& -1 & 0 &\dots   & \dots &\dots \\
                           -1 & 2 & -1 &\dots &\dots &\dots \\
                           & -1 & 2 & 0 & \dots & \dots \\
                           & \dots   & \dots &\dots   &\dots & \dots \\
                           &   &  &-1  &2& -1 \\
                           &    &  &   &-1 & 2 \\
                      \end{bmatrix}\begin{bmatrix}
                           v_1\\
                           v_2\\
                           \dots \\
                          \dots  \\
                          \dots \\
                           v_n\\
                      \end{bmatrix}
  =\begin{bmatrix}
                           g_1\\
                          g_2\\
                           \dots \\
                           \dots \\
                          \dots \\
                           g_n\\
                      \end{bmatrix}.
\]
For simplicity, $v_i$ is from now on written as $u_i$.
Furhter, the vectors $ \mathbf{a}, \mathbf{b}, \mathbf{c}$ are defined as the matrix-elements, there $ \mathbf{b}$ is placed along the diagonal, $\mathbf{a}$ is the lower diagonal and $\mathbf{c}$ is the upper diagonal. $g_i$ stands for $h^2f_i$, the solution for each equation. \\
\[
    \mathbf{A} = \begin{bmatrix}
                           b_1& c_1 & 0 &\dots   & \dots &\dots \\
                           a_1 & b_2 & c_2 &\dots &\dots &\dots \\
                           & a_2 & b_3 & c_3 & \dots & \dots \\
                           & \dots   & \dots &\dots   &\dots & \dots \\
                           &   &  &a_{n-2}  &b_{n-1}& c_{n-1} \\
                           &    &  &   &a_{n-1} & b_n \\
                      \end{bmatrix}\begin{bmatrix}
                           u_1\\
                           u_2\\
                           \dots \\
                          \dots  \\
                          \dots \\
                           u_n\\
                      \end{bmatrix}
  =\begin{bmatrix}
                           g_1\\
                           g_2\\
                           \dots \\
                           \dots \\
                          \dots \\
                           g_n\\
                      \end{bmatrix}.
\]
This linear set of equations can be solved by using the Thomas algorithm.
\paragraph{Thomas Algorithm}
The Thomas algorithm can be divided into two steps: a forward substitution and a backward substitution.\\
In the forward substitution, the subdiagonal elements are eliminated. Starting from the top, this is done by summing two successive appropriately scaled rows in the matrix. The mathematical expression for the new matrix elements is given as
\begin{equation*}
\begin{aligned}
\tilde{b}_i &= b_i - a_{j-1} \cdot c_{i-1}/ \tilde{b}_{i-1}  \hspace{0.5cm} \mathrm{for} \hspace{0.1cm} i=2,..., n-1 \\
\tilde{g}_i &= g_i - a_{j-1} \cdot \tilde{g}_{i-1} / \tilde{b}_{i-1}  \hspace{0.5cm} \mathrm{for} \hspace{0.1cm} i=2,..., n-1
\end{aligned}
\end{equation*}
The first row remains unchanged. \\
The backward substitution starts at the last row of the matrix. Here, the expression for $u_{n-1}$ becomes immediately clear
\begin{equation*}
\begin{aligned}
\tilde{b}_{n-1} u_{n-1} &= \tilde{g}_{n-1} \\
u_{n-1} &= \tilde{g}_{n-1} / \tilde{b}_{n-1}
\end{aligned}
\end{equation*}
Going up row for row, $u_i$ can be expressed as
\begin{equation*}
u_i = ( \tilde{g_i}-c_i u_{i+1} )/ \tilde{b_i} \hspace{0.5cm} \mathrm{for} \hspace{0.1cm} i=n-2,...,1 
\end{equation*}

\subparagraph{Special Case}
For our computation, we set identical matrix elements along the diagonal and identical (but different) values for the non-diagonal elements. More specific, we set $b_i = 2$ and $a_i = c_i = -1$. Thus the Thomas algorithm can be specialized to our specific case. It can be shown that the elements along the diagonal can be precalculated with
\begin{equation*}
\tilde{b_i} = \frac{i+1}{i}
\end{equation*}
That means that $\mathbf{\tilde{b}}$ no longer needs to be computed in the forward loop. \\
The forward loop is therefore given by
\begin{equation*}
\tilde{g}_i = g_i - a_{j-1} \cdot \tilde{g}_{i-1} / \tilde{b}_{pre, i-1}  \hspace{0.5cm} \mathrm{for} \hspace{0.1cm} i=2,..., n-1 \\
\end{equation*}
there $\tilde{b}_{pre}$ stands for the precalculation.
Similarly, the backward loop is now also computed with the precalculated $\mathbf{\tilde{b}}$
\begin{equation*}
u_i = ( \tilde{g_i}-c_i u_{i+1} )/ \tilde{b}_{pre,i} \hspace{0.5cm} \mathrm{for} \hspace{0.1cm} i=n-2,...,1 
\end{equation*}
The tables below show what effect this precalculation has on the number of FLOPs in the forward and backward substitution.

\begin{table}[ht]
\caption{FLOPS for general algorithm}
\centering
	\begin{tabular}{c c c c}
		\hline
		FLOPs   & Forward & Backward & Total \\ [0.5ex]
		\hline
		$\mathbf{g}$ & $3 (n-2)$       & $3 (n-2)$        & -    \\
        	$\mathbf{u}$ & $3 (n-2)$       & -        & -     \\
        	Total        & -       & -        &  $9 (n-2)$      \\ [1ex]
		\hline
	\end{tabular}
\end{table}	

\begin{table}[ht]
\caption{FLOPS for special algorithm}
\centering
	\begin{tabular}{c c c c}
		\hline
		FLOPs   & Forward & Backward & Total \\ [0.5ex]
		\hline
		$\mathbf{g}$ & $3 (n-2)$       & -        & -    \\
        	$\mathbf{u}$ & -       & $3 (n-2)$        & -     \\
        	Total        & -       & -        & $6 (n-2)$      \\ [1ex]
		\hline
	\end{tabular}
\end{table}	
XXXXXX comment on FLOPS

\paragraph{LU decomposition}
Another way to solve the set of linear equations is the LU decomposition.
This method factorizes the given square $\mathbf{A}$ matrix into two triangular matrices, one lower triangular matrix, $ \mathbf{L}$, and one upper triangular matrix, $\mathbf{U}$.
The original matrix $\mathbf{A}$ is then given by their matrix-product, e.g. $ \mathbf{A}= \mathbf{L} \mathbf{U}$.
The Gauss Elimination Method is used on $\mathbf{A}$ to form $\mathbf{U}$.
The diagonal elements of $\mathbf{L}$ er set to one. For a $3 \times 3$ matrix $\mathbf{A}$, the LU decomposition would look like
\[
     \begin{bmatrix}
                           a_{11}	& a_{12} & a_{13} \\
                           a_{21} & a_{22} & a_{23}  \\
                           a_{31} & a_{32} & a_{33}  \\
                      \end{bmatrix}
  =\begin{bmatrix}
                           1		& 0 & 0 \\
                           l_{21} & 1 & 0  \\
                           l_{31} & l_{32} & 1  \\
                      \end{bmatrix}
   \begin{bmatrix}
                           u_{11}& u_{12} & u_{13} \\
                           0 		& u_{22} & u_{23}  \\
                           0 		& 0 & u_{33}  \\
                      \end{bmatrix}
\]
The LU decomposition allows to rewrite the equation $\mathbf{Ax} = \mathbf{g}$ into two systems
\begin{equation*}
\begin{aligned}
\mathbf{Ly} &= \mathbf{g} \\
\mathbf{Ux} &= \mathbf{y}
\end{aligned}
\end{equation*}
Forward substitution can be used to solve the system $\mathbf{Ly} = \mathbf{g}$ and backward substitution can be used to solve $\mathbf{Ux} = \mathbf{y}$.
%https://www.geeksforgeeks.org/l-u-decomposition-system-linear-equations/
%http://www.math4all.in/public_html/linear%20algebra/chapter2.7.html
\paragraph{General Algorithm}
%http://folk.ntnu.no/leifh/teaching/tkt4140/._main040.html
Given below are pseudocodes for the Thomas algorithm and the LU decomposition.
\subparagraph{Thomas Algorithm}
\begin{enumerate}
\item Setting the input variables (x0, xn, a, b, c, n) \\
	x0 and xn confine the interval (here from 0 to 1). \\
	n stands for the number of integration points. \\
	In our case, the elements along the tridiagonal are constant, e.g. $a_i=a$, $b_i=b, c_i=c$ for all i. 
	
\item Initialization: Defining the step size h. \\
	Allocation of memory for $ \mathbf{a}, \mathbf{b}, \mathbf{c}, \mathbf{g}, \mathbf{exact}$  and filling them up with values.

\item Thomas algorithm with forward substitution: for (i=1, n-1)\\
	general case: update $\mathbf{b}, \mathbf{g}$ \\
	special case: update $\mathbf{g}$

\item Thomas algorithm with backward substitution (general + special case): for (i=-2,n) compute $\mathbf{u}$

\item Testing analytic solution: compute relative error $  \epsilon =  \left\lvert \frac{u_i-exact_i}{exact_i} \right\rvert$ 

\item Compute solution with LU decomposition

\item Measure CPU time for the different algorithms

\item Write results to file and if applicable plot them
\end{enumerate}

\subparagraph{LU decomposition}
Using armadillo 
\begin{enumerate}
\item Initialize Setting the input variables (x0, xn, a, b, c, n) \\
	x0 and xn confine the interval (here from 0 to 1). \\
	n stands for the number of integration points. \\
	In our case, the elements along the tridiagonal are constant, e.g. $a_i=a$, $b_i=b, c_i=c$. 

\item Setting up matrix $\mathbf{A}$ \\
	 Filling in values in for(1, 
		n-1 ???????????)
		
\item Compute $\mathbf{L}$ and $\mathbf{U}$ (built-in function)

\item Compute $\mathbf{y}$ in $\mathbf{Ly}=\mathbf{g}$ using forward substitution (built-in function)

\item Compute $\mathbf{x}$ in $\mathbf{Ux}=\mathbf{y}$ using backward substitution (built-in function)\\
		$\mathbf{x}$ is the solution-vector $\mathbf{u}$ 
\end{enumerate}
\section{Results}
First, the results for computing the general Thomas algorithm for the matrices of the size $10x10$, $100x100$, $1000x1000$ which corresponds to $n=10$, $n=100$ and $n=100$.
%\begin{figure}[H]
%\begin{center}
%\graphicspath{ {Oppg2/} }
%\includegraphics[width=10cm]{empRadius.png}
%\caption{Empirisk radius til Mo, Sn og Pb samt usikkerhet}
%\end{center}
%\end{figure}

\section{Conclusion}
\section{References}











% ------------------- end of main content ---------------

\end{document}

